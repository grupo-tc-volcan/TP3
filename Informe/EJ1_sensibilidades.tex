\subsection{An\'alisis de sensibilidades}

Para dicho an\'alisis se prosigui\'o mediante la ecuaci\'on \ref{sens}, siendo $X$ el valor de un componente e $Y$ el par\'ametro del circuito cuya sensibilidad respecto a $X$ se quiere calcular.

\begin{equation}
S^{Y}_{X}= \lim_{\Delta X \to \infty} \left(\frac{\Delta Y / Y}{\Delta X / X}\right) = \frac{X}{Y}\cdot\frac{dY}{dX}
\label{sens}
\end{equation}

Los par\'ametros sobre los que se analiz\'o la sensibilidad de cada uno de los componentes son $\omega_Z$, $\omega_P$ y $Q$. Sus resultados fueron los siguientes:

Sensibilidades de $\omega_Z$ respecto a cada componente del circuito:
\begin{equation}
	\begin{cases}
	S^{\omega_Z}_{R_4}= \frac{1}{2} \cdot \frac{R_{5}}{R_{4} + R_{5}}\\ \\
	S^{\omega_Z}_{R_5}= \frac{1}{2} \cdot \frac{R_{4}}{R_{4} + R_{5}}\\ \\
		S^{\omega_Z}_{C_2} = S^{\omega_Z}_{C_6}= S^{\omega_Z}_{R_1}=S^{\omega_Z}_{R_3}=S^{\omega_Z}_{R_8}    =- \frac{1}{2} 
	\end{cases}
\end{equation}

Sensibilidades de $\omega_P$ respecto a cada componente del circuito:
\begin{equation}
\begin{cases}
	S^{\omega_P}_{R_5} =	-  \frac{1}{2} \cdot \frac{R_{8}}{R_{5} + R_{8}} \\ \\
	S^{\omega_P}_{R_8} =	-  \frac{1}{2} \cdot \frac{R_{5}}{R_{5} + R_{8}}\\ \\
	S^{\omega_P}_{C_2} = S^{\omega_P}_{C_6}= S^{\omega_P}_{R_1}=S^{\omega_P}_{R_3}=	-  \frac{1}{2}\\ \\
	S^{\omega_P}_{R_4} = \frac{1}{2}
	\end{cases}
\end{equation}

Sensibilidades de $Q$ respecto a cada componente del circuito:
\begin{equation}
\begin{cases}
S^{Q}_{R_6} = \frac{R_{7}}{R_{6} + R_{7}}\\ \\
S^{Q}_{R_7} = \frac{R_{6}}{R_{6} + R_{7}}\\ \\
S^{Q}_{R_5} = \frac{1}{2} \frac{(C_{2} R_{1} R_{3} R_{5} R_{8})^2 \left(C_{6} R_{4} R_{5} - 1\right)}{C_{6} R_{4} \left(R_{5} + R_{8}\right) \left(R_{6} + R_{7}\right)^{2}} \\ \\
S^{Q}_{R_8} = \frac{1}{2} \frac{(C_{2} R_{1} R_{3} R_{5} R_{8})^2 \left(C_{6} R_{4} R_{8} - 1\right)}{C_{6} R_{4} \left(R_{5} + R_{8}\right) \left(R_{6} + R_{7}\right)^{2}} \\ \\
S^{Q}_{C_6} = S^{Q}_{R_4} =\frac{1}{2} \\ \\
S^{Q}_{C_2} = S^{Q}_{R_1} = S^{Q}_{R_3} =-\frac{1}{2}
\end{cases}
\end{equation}

A partir de las expresiones de las sensibilidades se puede decir que en caso de querer variar $\omega_Z$ lo conveniente es modificando $C_2$, $C_6$, $R_1$, $R_3$ o $R_8$ ya que tienen valores constantes. Al modificar el $\omega_Z$ , uno puede desplazar la frecuencia a la cual se encuentra el pico del Notch, ya que dicho pico se debe al Q grande del cero, y a la ubicaci\'on de dicho cero. Para el $\omega_P$ lo conveniente es variar $C_2$, $C_6$, $R_1$, $R_3$ o $R_4$. Y finalmente para el Q los mejores componentes de ajuste al valor deseado son $C_6$, $R_4$, $C_2$, $R_1$ o $R_3$. En m\'odulo todas presentan el mismo valor y al no depender de los valores de otros componentes, permiten hacer mejor un ajuste. 
