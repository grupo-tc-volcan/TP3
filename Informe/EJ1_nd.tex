\subsection{ ''Notch Depth''}

Se llama $notch depht$ a la profundidad del pico del filtro notch. Esta profundidad es la m\'axima atenuaci\'on que presentar\'a la se\~nal de entrada.

\paragraph*{C\'alculo te\'orico}

Observando la funci\'on transferencia ideal del circuito, que se transcribe a continuaci\'on:

\begin{equation}
H(s) = \frac{2}{1+k^2} \cdot \frac{s^2 + \left( \frac{k}{RC}\right)^2}{s^2 + \frac{1}{RCQ} s + \left(\frac{1}{RC}\right)^2}
\label{vovi_simple2}
\end{equation}

Se puede decir que dado que el Q del cero es $\infty$, el pico del notch llega a $-\infty$. Esto es para el caso ideal, el cual no se cumple al momento de medir. Para ver anal\'iticamente lo que ocurre con el pico del notch, a continuaci\'on se presenta la derivada de la transferencia obtenida, luego de haber reemplazado en la expresi\'on de la funci\'on transferencia con los valores de los componentes, se deriva la siguiente expresi\'on:

\begin{equation}
H(s) = 1,59 \cdot \frac{s^2+73,52 \cdot 10^6}{1,26 s^2 + 6666,6 s + 177,78 \cdot 10^6}
\end{equation}

\begin{equation}
H'(s) = \frac{\left(10,6 \cdot 10^3 s^{2} + 33,15 \cdot 10^7 s - 77,93 \cdot 10^{10}\right)}{s^{4} + 13,33 \cdot 10^3 s^{3} + 40 \cdot 10^7 s^{2} + 23,7 \cdot 10^11 s + 3.16 \cdot 10^{16}}
\label{deriv}
\end{equation}

Reemplazando con $s=j\cdot2\cdot \pi \cdot f$:

\begin{equation}
H'(f) = \frac{- 41,85  \cdot 10^4 f^{2} + 2083169982.40241 i f - 779304206880.0}{1558.54545654404 f^{4} - 3307303.10587019 i f^{3} - 15791507409.6255 f^{2} + 14893513515514.0 i f + 3.16057284 \cdot 10^{16}}
\end{equation}

Igualando la expresi\'on a cero, se obtienen sus dos extremos:

\begin{equation}
	\begin{cases}
	
	\end{cases}
\end{equation}

Para obtener el valor de la frecuencia del notch anal\'ticamente se obtuveo que la frecuencia a la cual se encuentra el pico del notch es a 1,082kHz.

\todo{alan}

Por otro lado, empleando el osciloscopio se observ\'o que la frecuencia para la cual la salida es m\'inima es 1,097kHz. Para dicho punto se obtuvo una atenuaci\'on de 35dB, la cual es menor que la predicha con los c\'alculos te\'oricos al no tratarse de un filtro notch ideal.

