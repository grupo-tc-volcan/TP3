\subsection{Funci\'on transferencia del circuito}
Debido a la complejidad del circuito \ref{circ1}, se  calcul\'o $\frac{V_O}{V_I}$ considerando a los dos amplificadores operacionales como ideales. Esto implica que se haya considerado $A_{vol}$ infinito. Para este an\'alisis, mediante resoluci\'on por nodos, se parti\'o de las relaciones entre corrientes entranes y salientes a cada uno de los tres nodos indicados en el circuito con tensi\'on $V_1$:

\begin{equation}
	\begin{cases}
		\frac{V_i - V_1}{(R_7 // \frac{1}{sC_6})} = \frac{V_1}{R_6} + \frac{V_1 - V_2}{R_1}\\ \\
		\frac{V_2 - V_1}{\frac{1}{sC_2}} = \frac{V_1 - V_O}{R_3}\\ \\
		\frac{V_O - V_1}{R_4} = \frac{V_1 - V_i}{R_8} + \frac{V_1}{R_5}
	\end{cases}
	\label{ecsbase}
\end{equation}

Operando algebraicamente, a partir de las ecuaciones \ref{ecsbase} se obtiene la funci\'on transferencia del circuito \ref{circ1}. Siendo $s = j\omega$, la misma se muestra a continuaci\'on:

\begin{equation}
 	H(s) = \frac{V_O}{V_I} = \frac{R_4 + R_5}{R_5} \cdot 
 	\frac
 	{ s^2
 		+\frac{R_4 R_6 R_8 + R_5 R_6 R_8 - R_4 R_5 R_7}{C_6 R_6 R_7 R_8 (R_4 + R_5)} \cdot s
 		+\frac{R_4 R_5}{C_2 C_6 R_1 R_3 R_8 (R_4 + R_5)}}
 	{s^2
 		+\frac{R_6 + R_7}{C_6 R_6 R_7} \cdot s
 		+\frac{R_4 (R_5 + R_8)}{C_2 C_6 R_1 R_3 R_5 R_8}
 	}
	\label{vovi}
\end{equation}


\subsubsection{Caracter\'sticas del circuito a partir de la funci\'on transferencia}

Si bien observando la funci\'on transferencia se puede decir que se trata de un filtro de segundo orden, a continuaci\'on se analiza m\'as en detalle a qu\'e tipo de filtro corresponde.

\paragraph*{Polos y ceros:}
A partir del t\'ermino de grado cero del numerador y del denominador del cociente de polinomios de la expresi\'on de $H(s)$ (Ecuaci\'on \ref{vovi}) se puede obtener la expresi\'on de los polos y ceros del circuito, ya que simb\'olicamente dichos t\'erminos son $\left(\frac{s}{\omega_Z}\right)^2$ y $\left(\frac{s}{\omega_P}\right)^2$respectivamente. Por lo tanto,

\begin{equation}
	\omega_Z = \pm\sqrt{\frac{R_4 R_5}{C_2 C_6 R_1 R_3 R_8 (R_4 + R_5)}}
	\label{ec_z}
\end{equation}

\begin{equation}
	\omega_P = \pm \sqrt{\frac{R_4 (R_5 + R_8)}{C_2 C_6 R_1 R_3 R_5 R_8}}
	\label{ec_p}
\end{equation}

Observando las diferencias entre los numeradores de las expresiones \ref{ec_z} y \ref{ec_p} se puede ver que como $R_5 < R_5 + 5_8$, el numerador del $\omega_Z$ es menor que el de $\omega_P$. Analizando los denominadores de igual manera, se v\'e que el denominador de $\omega_Z$ es mayor al denominador de $\omega_P$ ya que $R_4 + R_5 > R_5$. Por lo tanto, $\omega_Z < \omega_P$. Esto indica que el circuito \ref{circ1} es un filtro $High-Pass$ $ Notch$.


\paragraph*{Ganancia del circuito (G):}  La ganancia $G$ para el caso de un filtro $High-Pass$ $ Notch$ se obtiene como $\lim_{s\to\infty}H(s)$, estando $H(S)$ definida por la expresi\'on \ref{vovi} y as\'i es como resulta:
 \begin{equation}
	G = \lim_{s\to\infty}H(s) = 1 + \frac{R_4}{R_5} 
\label{G}
\end{equation}

\paragraph*{Factor de calidad (Q):} El coeficinete que acompa\~na a "s" en el denominador de la expresi\'on \ref{vovi} es $\frac{\omega_P}{Q}$. Por lo tanto, a partir de dicho coeficiente y de la expresi'on \ref{ec_p} correspondiente al $\omega_P$, se obtiene que:

\begin{equation}
	Q = \frac{R_{6} R_{7}}{R_{6} + R_{7}} \cdot \sqrt{\frac{C_{6} R_{4} \left(R_{5} + R_{8}\right)}{C_{2} R_{1} R_{3}R_{5} R_{8} }}
\end{equation}
	
	
	
	
	
	